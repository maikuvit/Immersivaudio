\documentclass[conference]{IEEEtran}
\IEEEoverridecommandlockouts
% The preceding line is only needed to identify funding in the first footnote. If that is unneeded, please comment it out.
\usepackage{cite}
\usepackage{amsmath,amssymb,amsfonts}
\usepackage{algorithmic}
\usepackage{graphicx}
\usepackage{textcomp}
\usepackage{xcolor}
\def\BibTeX{{\rm B\kern-.05em{\sc i\kern-.025em b}\kern-.08em
    T\kern-.1667em\lower.7ex\hbox{E}\kern-.125emX}}
\begin{document}

\title{Immersivaudio: music generation and enhancement for videos.\\
{\footnotesize \textsuperscript{*}Note: Sub-titles are not captured in Xplore and
should not be used}
\thanks{Instituto Superior Técnico, Universidade de Lisboa}
}

\author{\IEEEauthorblockN{Michele Vitale}
\IEEEauthorblockA{\textit{ist1111558}}
\and
\IEEEauthorblockN{Daniele Avolio}
\IEEEauthorblockA{\textit{ist1111559}}
\and
\IEEEauthorblockN{Teodor Chakarov}
\IEEEauthorblockA{\textit{ist1111601}}
}

\maketitle

\section{Introduction}

\section{Problem description}

\section{Proposed solution}

\section{Architecture}

\section{Deployment}

\section{Implemented features}

\section{Evaluation}

\section{Conclusions and future work}


\begin{thebibliography}{00}
    \bibitem{du2023conditional}
    Yuexi Du, Ziyang Chen, Justin Salamon, Bryan Russell, Andrew Owens,
    \emph{Conditional Generation of Audio from Video via Foley Analogies},
    \emph{arXiv preprint arXiv:2304.08490},
    2023.
\end{thebibliography}

\end{document}
