\documentclass[conference]{IEEEtran}
% The preceding line is only needed to identify funding in the first footnote. If that is unneeded, please comment it out.
\usepackage{cite}
\usepackage{amsmath,amssymb,amsfonts}
\usepackage{algorithmic}
\usepackage{graphicx}
\usepackage{textcomp}
\usepackage{xcolor}
\def\BibTeX{{\rm B\kern-.05em{\sc i\kern-.025em b}\kern-.08em
    T\kern-.1667em\lower.7ex\hbox{E}\kern-.125emX}}
\begin{document}

\title{Immersivaudio: audio generation based on video features.\\
{\footnotesize Group 01}
}


\author{\IEEEauthorblockN{Michele Vitale}
\IEEEauthorblockA{\textit{ist1111558}}
\and
\IEEEauthorblockN{Daniele Avolio}
\IEEEauthorblockA{\textit{ist1111559}}
\and
\IEEEauthorblockN{Teodor Chakarov}
\IEEEauthorblockA{\textit{ist1111601}}
}

\maketitle

\begin{abstract}
This document is a model and instructions for \LaTeX.
This and the IEEEtran.cls file define the components of your paper [title, text, heads, etc.]. *CRITICAL: Do Not Use Symbols, Special Characters, Footnotes, 
or Math in Paper Title or Abstract.
\end{abstract}

\begin{IEEEkeywords}
component, formatting, style, styling, insert
\end{IEEEkeywords}

\section{Introduction}
The topic of media generation has been exponentially growing during the last few years. Since the release of models and services based on state-of-art AI techniques, such as StableDiffusion and ChatGPT, it has been frequent to have media generative applications, with the most important part being that they can be easily accessed even by users that do not have competences and knowledge on Artificial Intelligence. 
Our proposal is a pipeline composed of different models, trained or open-wheigths, to generate audiovisive multimedia. 

\section{Problem description}
Main goal of our project is to provide a 
Che vogliamo fare 


\section{Problem importance}

Persone ipovedenti o cieche
Audio enhancer per contenuti (video di persona che parla, musica di sottofondo aggiunta in post)


\section{Architecture}
architecture schema picture


\subsection{Feature extraction}
video $->$ features extraction
\subsection{Prompt enhancer}
features $->$ well constructed prompt
\subsection{Audio generator}
prompt $->$ generated audio
\subsection{Reconstructor}
video + audio $->$ output multimedia


\section*{References}


\begin{thebibliography}{00}
\bibitem{b1} G. Eason, B. Noble, and I. N. Sneddon, ``On certain integrals of Lipschitz-Hankel type involving products of Bessel functions,'' Phil. Trans. Roy. Soc. London, vol. A247, pp. 529--551, April 1955.
\end{thebibliography}
\end{document}
